\chapter{Organisation du code }

\section{Packages}

\dirtree{%
.1 main.
.2 resources.
.2 takenoko.
.3 Deque.
.3 Irrigation.
.3 Joueur.
.4 Strategie.
.5 StrategieAction.
.5 StrategieAmenagement.
.5 StrategieCoord.
.5 StrategieIrrig.
.5 StrategieJardinier.
.5 StrategiePanda.
.3 Objectives.
.4 Amenagement.
.4 Patterns.
.3 Plot.
.3 Properties.
.3 Util.
.4 Comparators.
.4 Exceptions.
}


~\\
  Après de multiples ré-aménagements, nous sommes arrivé à cette structuration des packages car ...


\section{Interfaces}
Les principales Interfaces sont celle des Stratégies, chaque stratégie que ce soit StratégieAction, StratégieAmenagement, etc... ont chacune leur Interfaces car plusieurs implémentations avec des idées différentes sont présentes. Toutes ces interfaces permettant l'implémentation d'une stratégie complexe composée d'une stratégie de chaque catégorie. 

\section{héritage}
AA