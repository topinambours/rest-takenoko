\externaldocument{./string}
\externaldocument{./conception}
\chapter{Présentation du projet}


\section{Sujet}
Le projet consiste à réaliser en Java une version numérique du jeu Takenoko créé par Antoine Bauza. Version textuel n'étant pas destinée à être jouée par des êtres humains mais par des robots jouants de façon autonome. Dans le Takenoko nous allons endosser le rôle d'un jardinier japonais affairé à répondre au mieux aux requêtes du vénérable empereur. Pour ce faire, nous aurons, durant notre tour de jeu, deux choses à faire : Regarder quel temps il fait et effectuer deux actions parmi les possibilités suivantes : agrandir le jardin, irriguer, bouger le jardinier, bouger le panda. 

\section{Avancement du projet}
Le projet en l'état est pratiquement complet, il manque en effet le dès qui pourra être implémenter ultérieurement car nous avons à l'heure actuelle privilégié l'injection de dépendance au dès.  

\section{Problématique soulevée}
Plusieurs problématiques ont été soulevées durant le projet qui sont les suivantes :

\subsection{Le bon partitionnement du code}
L'organisation du code a été pour nous un assez grand problème notamment à cause de l'apparition de "God classe"\footnotemark\\ comme la classe de Jeu qui concentré un trop grand nombre de fonctions. Nous avons donc fait plusieurs ré-aménagements (Voir paragraphe Conception\ref{conception}) 

%Contenu de la note précédemment marquée avec \footnotemark
\footnotetext{La "God classe" est une classe comportant un très grand nombre de fonctions qui pourraient être dispatcher dans d'autre classes}

\subsection{L'injection de dépendance}

Du fait de la nouveauté de l'outils, nous avons rencontrer plusieurs problèmes lors de l'injection de dépendance. (Voir paragraphe Spring\ref{spring})
