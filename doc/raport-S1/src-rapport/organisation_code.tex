\chapter{Organisation du code }

\section{Packages}

\begin{minipage}{0.45\textwidth}
\dirtree{%
.1 main.
.2 resources.
.2 takenoko.
.3 Deque.
.3 Irrigation.
.3 Joueur.
.4 Strategie.
.5 StrategieAction.
.5 StrategieAmenagement.
.5 StrategieCoord.
.5 StrategieIrrig.
.5 StrategieJardinier.
.5 StrategiePanda.
.3 Objectives.
.4 Amenagement.
.4 Patterns.
.3 Plot.
.3 Properties.
.3 Util.
.4 Comparators.
.4 Exceptions.
}
\end{minipage}
\begin{minipage}{0.45\textwidth}
  Après de multiples réaménagements, nous sommes arrivé à cette structuration des packages.\\
  
\end{minipage}

\section{Interfaces}
Les principales interfaces sont celle des Stratégies, chaque stratégie que ce soit Stratégie action, Stratégie Aménagement, etc. ont chacune leur interface car plusieurs implémentations avec des idées différentes sont présentes. Toutes ces interfaces permettant l'implémentation d'une stratégie complexe composée d'une stratégie de chaque catégorie. 

\section{Héritage}
Nous utilisons l'héritage pour les éléments ayant comme dits précédemment des besoins fonctionnels similaires, nous avons donc du code générique pour les pioches et les cartes objectives. Les comportements du type piochent/validation d'une carte étant commun à l'ensemble des cartes, il est important que les classes spécifiques héritent d'une classe contenant les comportements primaires de ces éléments.